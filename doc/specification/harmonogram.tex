\begin{comment}

----------------------------------------------------------------------------------------------------

zadania programistyczne:
    ściągacz
    parser
    przechowywajka danych
    połączenie komputer-smartphone
    gui android
    gui desktop
    wykresy
    *średnia krocząca
    *integracja z g+

zadania projektowe:
    format pliku z danymi programu
    projekt gui android
    projekt gui desktop
    protokół wymiany danych
    
zadania inne:
    specyfikacja
    diagramy uml

----------------------------------------------------------------------------------------------------

+   - do beta

[1: fundament]  (2 tyg)
projektowanie:  format pliku z danymi programu  (< 1 tyg)   AR   +
implementacja:  ściągacz                        (< 1 tyg)   A    +
implementacja:  parser                          (< 1 tyg)   A    +
implementacja:  przechowywajka danych           (< 1 tyg)   R    +

[2: android-gui]    (4 tyg) A
projektowanie:  gui android (1 tyg) +
implementacja:  gui android (3 tyg) +

[3: desktop-gui]    (4 tyg) R
projektowanie:  gui desktop (1 tyg) +
implementacja:  gui desktop (3 tyg) +

[4: wykresy]    (1 tyg)
projektowanie:  rysowanie wykresów  (< 1 tyg)   A   +
implementacja:  wykresy android     (< 1 tyg)   A   +
implementacja:  wykresy desktop     (< 1 tyg)   R   +

[5: synchronizacja] (2 tyg)
projektowanie:  protokół wymiany danych                 (< 1 tyg)   R
implementacja:  połączenie komputer-smartphone          (< 1 tyg)   AR
implementacja:  przesyłanie danych komputer-smathphone  (< 1 tyg)   AR

[6: dodatkowe]  (2 tyg)
implementacja:  średnia krocząca        (< 1 tyg)   A
implementacja:  integracja z g+         (2 tyg)     AR
implementacja:  informacje o firmach    (2 tyg)     R

----------------------------------------------------------------------------------------------------
\end{comment}

\begin{table}[H]
\centering
\label{harmonogram}
    \begin{tabular}{|p{5cm}|c|c|c|c|c|}
    \hline
        \textbf{zadanie} & \textbf{czas trwania} & \textbf{rozpoczęcie} & \textbf{zakończenie} & \textbf{osoba}\\
    \hline
        usecase'y & 2 tyg. & 10.10.2013 & 24.10.2013 & Joanna, Rafał\\
        diagramy UML klas & 3 tyg. & 17.10.2013 & 07.12.2013 & Joanna, Rafał\\
    \hline
        projektowanie: format pliku z danymi & $<$ 1 tydz. & 21.10.2013 & 28.10.2013 & Joanna, Rafał\\
        implementacja: klasa pobierająca dane & $<$ 1 tydz. & 28.10.2013 & 31.10.2013 & Joanna\\
        implementacja: klasa przetwarzająca dane & $<$ 1 tydz. & 31.10.2013 & 04.11.2013 & Joanna\\
        implementacja: klasy przechowujące dane & $<$ 1 tydz. & 28.10.2013 & 04.11.2013 & Rafał\\
    \hline
        projektowanie: interfejs graficzny dla Androida & 1 tydz. & 04.11.2013 & 11.11.2013 & Joanna, Rafał\\
        projektowanie: interfejs graficzny dla desktop & 1 tydz. & 04.11.2013 & 11.11.2013 & Rafał\\
        implementacja: interfejs graficzny dla Androida & 3 tyg. & 11.11.2013 & 02.12.2013 & Joanna\\
        implementacja: interfejs graficzny dla desktop & 3 tyg. & 11.11.2013 & 02.12.2013 & Rafał\\
    \hline
        projektowanie: rysowanie wykresów & $<$ 1 tydz. & 02.12.2013 & 05.12.2013 & Joanna, Rafał\\
        implementacja: wykresy na Android & $<$ 1 tydz. & 05.12.2013 & 09.12.2013 & Joanna\\
        implementacja: wykresy na desktop & $<$ 1 tydz. & 05.12.2013 & 09.12.2013 & Rafał\\
    \hline
        \multicolumn{5}{|c|}{wersja beta}\\
    \hline
        projektowanie: protokół wymiany danych & $<$ 1 tydz. & 09.12.2013 & 12.12.2013 & Joanna, Rafał\\
        implementacja: połączenie komputer - smartphone & $<$ 1 tydz. & 12.12.2013 & 19.12.2013 & Joanna, Rafał\\
        implementacja: przesyłanie danych & $<$ 1 tydz. & 19.12.2013 & 23.12.2013 & Joanna, Rafał\\
    \hline
    \end{tabular}
\end{table}

